{\it Message from the Director}


{\bf Fractal Academics:}
	
At IITH we are exploring a new approach to structuring the academic curriculum. We grapple with many questions in educating our students:  lack of motivations among students, low attendance and uneven student interest.  Other challenges include:  the gulf between theory and practice, breadth vs. depth, and the relevance of non-core subjects. How do we tailor the curriculum based on individual potential? How do we make the curriculum interdisciplinary? How do we increase industry interactions? How do we incorporate research in under graduate curriculum? This list of questions is by no means exhaustive, and each question, perhaps, begets a thesis. 

In order to provide some resolution to these difficult questions, we first started with fractional credit courses. A typical 3-credit course has 42 contact hours; we developed courses with 0.5, 1.0, 1.5, 2.0, 2.5, and 3 credits having 7, 14, 21, and 28, 35 and 42 contact hours. The motivation was to atomize the teaching program and also involve industry partners in some aspects of academics.  The student enthusiasm, their commitment, and their output was very high in these courses. Based on the overall success of fractional credit courses, we developed a complete 4-year curriculum, referred to as Fractal Academics.  The core of fractal academics is that breadth courses are of 1 credit, while depth courses are typically of 1.5 to 2.5 credits. In essence, we are atomizing the academic program, providing a more holistic education, and in the long run giving students the choice to design their curriculum.

Fractal Academics was first implemented at IITH for the Electrical Engineering Department in Aug 2013. From Aug 2014 all engineering departments followed Fractal Academics. In 2016 Aug, Fractal Academics was also implemented at IIT Bhilai in Computer Science and Engineering, Electrical Engineering and Mechanical Engineering. 

Fractal Academics was developed by the faculty of IITH and it is to them the novel academic program owes its success. Also, the students of IITH deserve a very special thanks for experimenting with and accepting this novel program.

Fractal Academics is constantly evolving based on feedback from students and faculty. We believe that it should evolve continuously and keep pace with changing times and changing aspirations of the students.  
